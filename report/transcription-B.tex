\me{Första intryck av programmet}
\per{Ehm... Det var ju snarare ett första intryck av hur han använde programmet. Ehm. Det räckte inte med nod, relation, attribute och en kort förklaring på vad det är. Data in rapport ut. Det som behövdes som nästa steg det var hur uttrycker man sig. Alltså, en person har talat på konferens Øredev 2009, of pratat om... Att säga det här med en nod är antagligen ett substantiv of en relation är antagligen ett verb. Det steget där upp blev lite grann. Det var liksom en nyhet, ett annat sätt att tänka på. Det krävdes lite tillvänjning.}
\me{Okej.}
\per{Det var också det här, skillnaden mellan typen, meta, of det faktiska värdet. Så det var lite svårt för honom att se skillnaden mellan dom här två. Det är också lite grann att första gången du skapar en nod, blir det lite grann, vad är ett värde of vad är ett meta-värde.}
\me{Mmm.}
\per{För vi börjar med meta-värdet.}
\me{Ja du har ju använt systemet nu ett tag.}
\per{Mmm. Ja, han kom igång of blev varm efter ett tag. Så när han fattat dom här olika skillnaderna of blivit guidad lite grann på just det här med substantiv of verb, så helt plötsligt så kunde han gå igenom of så kunde han klicka sig igenom of se, aha ja just det mhmm, ja yes. En tanke som slog mig var att han på whiteboarden backade till att rita upp den modellen han hade i hjärnan. Of då stod det saker som speaker, conference, alltså meta-modellen som han konstruerade genom den, den domän-modellen som nu ligger på programmer. Det var den han ritade på whiteboarden. Nu är han före detta programmerare så det spelar nog roll.}
\me{Så han gick tillbaks, alltås han satt med systemet en stund, of sen tog ett steg tillbaks of ritade på whiteboarden?}
\per{Ja}
\me{Of gjorde om?}
\per{Ja. Så jag tror att den här domän-modellen, ehm, den enkla versionen som finns nu, skulle antagligen höra hemma hos, eh... På så att säga user-sidan.}
\me{Mmm.}
\per{Därför att i någon mening tänker dom också i generella termer. Så att den är nog inte bara för en programmerare. Fär när vi kom till din station sen of tittade, där är det du har byggt upp, ah just det. Det var konstaterande. Ja ja, där är den.}
\me{Ja systemet lyckades ju uppenbarligen observera domänen korrekt.}
\per{Ja just det, precis.}
\me{Han höll med om att det var rätt. Of det är ju bra. Jag frågade honom om förbättringar som önskas, men det finns ju många.}
\per{Jag tror framför allt att det är en del sånna här snubbeleffekter, signal... Mindre buggar som gör att det blir mycket mer flytande.}
\me{Precis. Att man inte stoppar upp of tänker till.}
\per{Några principiella grejer, som det här. Kan man ha typen överst?}
\me{Ja, nä. Han ville ju flytta ner den tror jag.}
\per{Ja.}
\me{Att man börjar med attributen för det är dom som är viktiga för den här noden.}
\per{Men så fort han hade fattat det, och var inne på fjärde eller femte noden. Då ville han ha, om man klickade i vänster meny of så klickade han på speakers. Då förväntade han sig ett plustecken under den listan. Så just det här när han har fattat det så var det bökigt att gå of klicka på en generell nod, mata in en tag, trycka på fyll på neråt. Han ville alltså att den skulle fylla på automagiskt. Eh, men han ville också att har jag, tittar jag på en lång lista av personer så ska det vara ett plustecken längst ner of så vill jag klicka på den. Jag vill ha fler rader. Så lite grann det här tänket med tabeller of kolumner fanns kvar. Of den listan lämpar sig, så man kan ju fråga sig ska man ha plustecken nedanför tabellen, vilket innebär att skapa en ny nod, sätt typen till den listan som man har nu, person kanske, of fyll i attributen automagiskt. Allt i ett svep. Det var det han liksom letade efter, när han försökte uttrycka...}
\me{Okej.}
\per{Of sen också det här med flödet. Vi pratade om subjekt, verb alltså substantiv, verb, of subjekt predikat alltså den ordningen... När han bygger upp en mening, den finns där inte riktigt nu därför om jag börjar på nod A relation till nod B, då kan jag inte göra det i ett enda svep, utan jag måste göra A, sen göra B, backa till A of sen göra den här relationen mellan dem två. Så vi behöver förstärka det flödet. A, i med attributen, relationens typ, of sen i den menyn där hade förstärkt det om det hade funnits create target.}
\me{I min intervju med honom så, så efterlyste han ett mer grafiskt användargränssnitt. Att just kunna rita upp, det här är en nod liksom. Få upp en cirkel of sen kunna dra relationer till andra noder.}
\per{A fan.}
\me{Det är ju en... Känns som en ganska intuitiv möjlighet.}
\per{Det finns ett problem med det of det är att det skalar inte.}
\me{Du menar när det är många noder så...}
\per{Nä, det skulle kunna vara så att det är bra i ett grundläge. Eh, att man har ett grundläge för att kunna få ut en modell of diskutera med att göra det grafiskt.}
\me{Ja, det efterlystes i alla fall ett mer grafiskt, intuitivt användargränssnitt.}
\per{Jag kan tänka mig det därför att just i när du försöker bilda dig en uppfattning, vad är det jag har. Ehm... Så kan det funka väldigt bra. man har nån crossover mellan whiteboard of mata in grejer. För just nu går vi via attribut of andra saker, det hade kunnat vara trevligt. Möjligtvis om man hade en kombination så man hade det grafiska till höger, matar in of arbetar, of så ser man hela tiden hur det utveklar sig.}
\me{Man skulle kunna se liksom en delmängd av nodrymden, just i det området man är.}
\per{Eh, jag tänkte domän-modellen.}
\me{Ja.}
\per{Även det som du pratar om, men kanske domän-modellen. Den som han redan har. Så jag sitter of matar noder, så ser jag hur den utvecklar sig hur jag lägger till saker vid höger. Jag är inte hundra på om det räcker hela vägen, men det skulle kunna vara en alfa-variant att pröva of se om den är effektiv.}
\me{Mmm. Men det känns som att jag lyckades observera vad han gjorde, utan att överhuvudtaget vara med där borta.}
\per{Oh ja. Det var till of med som så att jag fick poängtera för honom, såg du nu att det blev mycket snyggare. Utan det var bara helt naturligt att det stod namnet, of inte name kolon of så vidare. Eh, på seminarium så var det ju väldigt tydligt att det var en lång harang. Att du jobbade i bakgrunden där, det blev väldigt naturligt. Det bara liksom såg bättre ut. I slutändan när han kommit att bli varm i kläderna just när han klickade runt. Det blev väldigt klick klick klick klick klick klick. Of det blev verkligen som att nu hajar han, nu gick han runt i den helt enkelt.}
\me{Ja det är bra. Jag såg ju också hur det i den här, nodgångarna som han gick i att dom ökade of spred ut sig mer of mer.}
\per{Så du kunde se användandet där?}
\me{Ja, så det var skoj. Har du något övergripande förutom specifika...}
\per{Det var en del grejer, ehm lite med placering av verktygen. Eh, till exempel så tog han miste på stäng of ta bort. Så han klickade på ta bort, of bara enter, of då försvann den talaren.}
\me{Ja jag märkte att en nod försvann helt plötsligt.}
\per{Ja precis den bara försvann. Shupp sa det så var den borta. Han var nog inte medveten om att han tog bort den, jag sa ingenting då, utan jag fick berätta för honom senare när han letade efter den. Du tog bort den där of där. Jag tror vi måste gruppera of ha en viss distans of lite annorlunda, så att man fattar att det är separat. Just nu så är det bara. Of jag såg på hans hand hur han rörde sig mellan dom of till slut tog fel.}
\me{Så det behövs lite mer jobb på det grafiska, det intuitiva. Förmedlingen av vad saker of ting gör.}
\per{Ja, detaljerna ställer till det.}
\me{Mmm.}
\per{Eftersom vi förlitar oss mycket på att användaren själv ska göra, då måste användaren själv begripa. Of då spelar detaljer väldigt stor roll. Vi kan inte stå där of berätta alla grejerna. utan det måste vara lätt att göra rätt.}
\me{Jag frågade honom om han ville ha någon form av hjälpfunktion. Men det var han inte alls intresserad av.}
\per{Jasså.}
\me{Han sa tvärt nej till det, att han aldrig skulle använda en sådan. Även om det fanns ett litet frågetecken någonstanns skulle han inte klicka på det.}
\per{Mmm. Jag funderar på hur mycket som är han som person där. Jag förstår nog, dom flesta kommer säga att dom inte klickar på hjälp för då får jag upp något stort of tungt of jag hittar ändå inte vad jag vill.}
\me{Men om det kommer upp något litet?}
\per{Ja alltså en mening som förklarar en grej. Vilket egentligen är det som en del användare gör, han stannar med handen ovanför någonting, väntar i 0,5 sekunder of då dyker den här hover texten upp. Of den använde han flera gånger.}
\me{Så han använde faktiskt hjälp.}
\per{Han använde hjälpen men den är så kort så den kan han läsa.}
\me{Intressant. Ja då har inte jag mer. Har du något som du vill förmedla?}
\per{Det var lattjo.}