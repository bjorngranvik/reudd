\me{Du har nu använt systemet.}
\per{Ja.}
\me{Efter en kort intro. Kände du att intron var tillräcklig?}
\per{Eh... Ja, det gjorde den. Det var den. I och med att jag har lite erfarenhet ifrån programmeringssidan då.}
\me{Ja precis.}
\per{Och det är ganska lätt att koppla det till eh... objekt.}
\me{Mmm. Okej, vad bra.}
\me{Kom du igång snabbt med systemet?}
\per{Ja det var ju lite, man glömde bort lite i början, man ändra sig. Attributen och så vidare. Men det var lättanvänt systemet.}
\me{Ja, vad bra. Jag har försökt göra mycket med eh... Eller jag har inte koncentrerat mig specifikt på användarvänlighet, men ändå försöker att ehm, att hålla det användarvänligt för att kunna öka upplevelsen... av det.}
\me{Var det nån gång som du stötte på liksom problem? Att det tog helt stopp och att... hur gör man nu?}
\per{Nej, det var ganska självinstruerande, systemet.}
\me{Okej.}
\per{Om vi bortser från alla buggar som var då. Så tittar bara på GUI:t så var det ganska enkelt då. Att förstå sig på.}
\me{Okej.}
\me{Ja du byggde upp ett system, och sen så visade jag dig en domän-modell här. Är det något som du kan relatera till? Känner du att det är den här domänen du har byggt upp?}
\per{Ja, det stämmer ganska bra överens.}
\me{Mmm.}
\per{Ja.}
\me{Märkte du att jag gjorde några förbättringar? Under tiden?}
\per{Nej}
\me{Det märktes inte?}
\per{Nej.}
\me{Okej.}
\per{Möjligtvis rapporterna var ju det som du la till då, under tiden. Mellan testpassen då.}
\me{Ja den märktes ju att den gick från att inte fungera till att fungera.}
\per{Ja, ja.}
\me{Dom förbättringarna jag gjorde var ju att jag ändrade representationen från den här default grejen där den skriver ut typen och sen så alla attribut.}
\per{Ja, ja nu ser jag det i efterhand ja. Det tänkte jag inte på, jag bara...}
\me{Det var bara... Var det naturligt liksom att det skulle...}
\per{Jaja.}
\me{Det borde se ut så här kanske, nåt sånt.}
\per{Ja, ja.}
\me{Du eh, du eh gjorde systemet på ett ann... ett nytt sätt, som jag inte har sett innan.}
\per{Okej.}
\me{Utan innan så har det ju varit att man har skapat många noder av samma typ. Du har ju, varje typ har bara en nod.}
\per{Ja.}
\me{Så då märker man inte dom här förbättringarna. Om du till exemple nu går in och vill lägga till en ny, ja vad har vi, person. Så kommer namn och personnummer automatiskt att hamna här.}
\per{Ah, okej.}
\me{Det var också en förbättring som jag gjorde, men den märktes ju då självklart inte, eftersom det inte las till fler.}
\per{Nej.}
\me{Ehm. Så det är ju mycket möjligt att min tanke om hur, hur användaren använder systemet är felaktig då.}
\per{Nej, det kan vara så att användaren tänker på ett helt annat sätt.}
\me{Ja.}
\per{Eh, dom kanske ser att eh... bara för man i det här exemplet då, träningskompis är inte samma sak som jag själv.}
\me{Nej okej.}
\per{Utan dom får en helt annan visuell bild över systemet.}
\me{Mmm.}
\per{Och kanske inte kopplar samman att person och person är samma.}
\me{Nej, för det här systemet är ändå tänkt att kunna användas av vilken användare som helst, så då måste det ju även kunna täcka in sånt här användande.}
\per{Precis. Så dom kanske ser en chef, och en anställd som två olika personer. Fast egentligen så är det...}
\me{Två olika typer då eller?}
\per{Två olika typer ja.}
\me{Känner du att det här typ-begreppet är lite förvirrande?}
\per{Ja du bör ändra namnet noder i alla fall.}
\me{Okej.}
\per{Om du ska titta på den häringa, Greta 52, eller vad hon kan heta. Så eh... nått annat enklare namn för användaren då.}
\me{Precis. Okej.}
\me{Lite mer att eh... språket blir mer allmänt?}
\per{Ja.}
\me{Inte så programmeringsinriktat?}
\per{Nej, precis. Lägg upp det ett par abstraktionsnivåer.}
\me{Ehm... Nån hjälpfunktion någonstanns. Tror du det hade varit någonting?}
\per{Ja, hjälpfunktion brukar alltid hjälpa. Det är därför dom heter hjälpfunktion... Nej, egentligen inte. Utan det är att man får sitta och eh... Det är så ganska enkelt ändå, så har man fått en, ett par träningsrundor så har man fattat konceptet.}
\me{Det är bättre att ha dom här, den lilla hjälpen man får om man har, när man kör mouse-over till exempel. Så får man en liten titel.}
\per{Ja precis, det är mycket bättre.}
\me{Okej.}
\me{Vad har du för övergripande intryck av idén, liksom tanken bakom?}
\per{Ehm. Jag tror den kan bli ganska bra, tanken bakom. Idén. För ofta så brukar man alltid stå där och så brukar man prata, hur vi ska ha systemet. Men här går man in och så visar man, hur det så här kopplas ihop allting. Och man får en ganska snygg modell utav det.}
\me{Ja det är ju väldigt enkelt, för nu hade vi ju. Vi hade ju inget samtal innan om vad det var för system som du skulle bygga. Men jag kunde ju direkt se vad det handlade om ändå. Och nu kanske man inte ska separera programmeraren och användaren så mycket, men det går i alla fall att göra det.}
\per{Ja.}
\me{Sen kan man ju flytta dom närmare. Dom kan ju sitta, användaren kan sitta bredvid och använda, och programmeraren kan titta både på själva, när dom sitter och använder och på vad dom gör i databasen.}
\per{Ja precis. Ett utmärkt exempel på hur man kan utvinna kraven hade jag, systemfunktion i alla fall.}
\me{Ja. Har du något annat du vill tillägga?}
\per{Nej, ser ganska bra ut. Gör om, fixa iordning buggar och så.}
\me{Buggfixa och lite mer användarvänlig?}
\per{Ja precis. Ett par releaser till så ska det nog gå igång ordentligt.}
\me{Ja vad bra. Då tackar jag så mycket!}
\per{Varsågod!}
